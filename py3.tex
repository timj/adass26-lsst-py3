% This is the aspauthor.tex LaTeX file
% Copyright 2014, Astronomical Society of the Pacific Conference Series
% Revision:  14 August 2014

% To compile, at the command line positioned at this folder, type:
% latex aspauthor
% latex aspauthor
% dvipdfm aspauthor
% This will create a file called aspauthor.pdf.

\documentclass[11pt,twoside]{article}
\usepackage{./asp2014}

\aspSuppressVolSlug
\resetcounters

\bibliographystyle{asp2014}

\markboth{Jenness et al.}{LSST and Python 3}

\begin{document}

\newcommand{\arxiv}[1]{\href{http://arxiv.org/abs/#1}{arXiv:#1}}
\title{Porting the LSST Data Management Pipeline Software to Python 3}
\author{Tim~Jenness$^1$
\affil{$^1$Large Synoptic Survey Telescope, Tucson, AZ, USA; \email{tjenness@lsst.org}}}

% This section is for ADS Processing.  There must be one line per author.
\paperauthor{Tim~Jenness}{tjenness@lsst.org}{0000-0001-5982-167X}{LSST}{Data Management}{Tucson}{AZ}{85719}{USA}

\begin{abstract}
The LSST data management science pipelines software consists of more than 100,000 lines of Python 2 code.
LSST operations will begin after support for Python 2 has been dropped by the Python community in 2020, and we must therefore plan to migrate the codebase to Python 3.
During the transition period we must also support our community of active Python 2 users and this complicates the porting significantly.
We have decided to use the Python \texttt{future} package as the basis for our port to enable support for Python 2 and Python 3 simultaneously, whilst developing with a mindset more suited to Python 3.
In this paper we report on the current status of the port and the difficulties that have been encountered.
\end{abstract}

\section{Background}

The Large Synoptic Survey Telescope \citep[LSST;][]{2008arXiv0805.2366I} will take about 15\,TB of image data per night and after ten years of operations will have 15\,petabytes of catalog data for the final data release and 0.5\,exabytes of image data\footnote{\url{http://lsst.org/scientists/keynumbers}}.
As part of the LSST Data Management System \citep{2015arXiv151207914J}, we are writing a suite of software packages to enable these data products to be created with sufficient quality and performance to meet the established science goals \citep{2009arXiv0912.0201L}.

Development of the LSST pipeline software began in 2004 \citep{2004AAS...20510811A}, when Python was at version 2.3, and has continued through the research and development phase \citep{}into the construction phase.

\section{Supporting Python 3}


\section{The Experience}




\acknowledgements The LSST software stack
is the result of the efforts of the many people who are part of the
Data Management Team at LSST, as well as outside contributors.  This
material is based upon work supported in part by the National Science
Foundation through Cooperative Support Agreement (CSA) Award
No. AST-1227061 under Governing Cooperative Agreement 1258333 managed
by the Association of Universities for Research in Astronomy (AURA),
and the Department of Energy under Contract No. DE-AC02-76SF00515 with
the SLAC National Accelerator Laboratory.  Additional LSST funding
comes from private donations, grants to universities, and in-kind
support from LSSTC Institutional Members.

\bibliography{py3}  % For BibTex

\end{document}
